\documentclass[12pt]{article}

\usepackage[utf8]{inputenc}
\usepackage[french]{babel}
\usepackage[T1]{fontenc}

\usepackage{graphicx}
\usepackage{hyperref}
\usepackage{amsmath}
\usepackage{amssymb}
\usepackage{amsthm}

%Ces commandes vont permettre d'afficher de belles définition et de belles propositions.

% \newtheorem{defgalien}{Définition}
% \theoremstyle{definition}

% Les commandes suivantes permettent de définir des raccourcis :
% \newcommand{\E}{\mathrm{E}}
% \newcommand{\R}{\mathbb{R}}
% \renewcommand{\P}{\mathbb{P}}

\title{Erratum du document sur les matrices}
\author{Fabien Delhomme}

\begin{document}

\maketitle

\section{Page 26}

Le tableau qui présente la suite de Fibonacci est malheureusement faux. Voici
les vraies premières valeurs de la suite de Fibonacci.
\begin{table}[h]
    \centering
    \begin{tabular}{|c|c|c|c|c|c|c|c|c|}
        \hline
        $n$ & 0 & 1 & 2 & 3 & 4 & 5 & 6 & 7 \\
        \hline
        $\mathcal{F}_n$ & 0 & 1 & 1 & 2 & 3 & 5 & 8 & 13  \\
        \hline
    \end{tabular}
    \caption{Liste des premières valeurs de la suite de Fibonacci}
\end{table}

\end{document}
